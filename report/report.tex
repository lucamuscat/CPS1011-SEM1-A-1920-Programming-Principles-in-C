
% Fix formatting errors in class jsonWriter pg114 lst 41
%fix formatting listing 40
%should code also have onehalfline spacing?

\documentclass[a4paper, 12pt, titlepage]{article}

%---------------Preamables---------------
\usepackage[none]{hyphenat}
\usepackage{tikz}
\usepackage{aeguill}
\usepackage{setspace}
\usepackage{listings}
\usepackage[newfloat]{minted}
\usepackage[justification=centering]{caption}
\usepackage[margin=0.78in]{geometry}
\usepackage{titlesec}
\usepackage{graphicx}
\usepackage{float}
\usepackage{multirow}
\usepackage{tabularx}
\usepackage{longtable}

\setlength{\footnotesep}{\baselineskip}

\newcolumntype{L}[1]{>{\raggedright\arraybackslash}p{#1}}
\newcolumntype{C}[1]{>{\centering\arraybackslash}p{#1}}
\newcolumntype{R}[1]{>{\raggedleft\arraybackslash}p{#1}}
\newenvironment{code}{\captionsetup{type=listing}}{}
% \sourcecode{Directory/NameOfFile}{Caption}{Label}
% #1: Code Snippet
% #2: Caption
% #3: Reference Label
\newcommand{\sourcecode}[3]{
    \begin{code}
      \inputminted[linenos,numbersep=5pt,gobble=0,frame=lines,framesep=2mm,]{c}{#1}
        \caption{#2}
        \label{lst: #3}
    \end{code}
}
% #1 - Scale
% #2 - Image path
% #3 - Caption
% #4 - Reference Label
\newcommand{\image}[4]{
  \begin{center}
  \includegraphics[scale=#1]{#2}
  \end{center}
  \captionof{figure}{#3}
   \label{fig: #4}
}
%---------------  End   ---------------

\begin{document}

%---------------Header---------------
%Good Syntax styles: trac, abap
\title{C Assignment Report}
\author{Luca Muscat}

\titleformat{\section}{\Large\bfseries\filcenter}{}{0em}{}
\titleformat{\subsection}{\normalsize\bfseries\filcenter}{}{0em}{}
\titleformat{\subsubsection}{\small\bfseries}{}{0em}{}
\maketitle

\pagebreak
\tableofcontents
\pagebreak
%---------------   End  ----------------

%--------------- Content ---------------
\begin{onehalfspacing}
  \section{Task 1}
  \label{sec:task1}
  In this task, a number of array-manipulating functions will have to be implemented. The functions which were implemented are as follows:

  \begin{enumerate}

  \item generate(): A function that populates an array with a
      sequence of N integers, in ascending order, and starting off integer i
      having a minimum value of 1. \emph{i} should be accepted as a function
      argument while N is defined as a constant. This function is
      destructive in the sense that it overwrites any previously generated
      values.
    \item shuffle(): A function that shuffles the items of an array
      argument and which makes use of stdlib.h’s rand() function.
    \item sort(): A function (implemented from scratch) that returns a
sorted array passed as an argument
\item shoot(): A function that zeros out one element from,
  a possibly unsorted, array at random. This function returns an error
  if at least one element had already been previously zeroed out.
\item target(): A function that returns the number
  (i.e. the actual value and not the array offset) that was zeroed out
  by a single call to shoot().

  \end{enumerate}
  \pagebreak
  \subsection{Task 1(a)}
  In this task, the functions listed in section \ref{sec:task1} will
  be implemented. For the sake of readability, the functions were split
  into two files which are the ``functions.h'' (which holds the function
  prototypes and their description via comments) file and
  ``functions.c'' which contains function's implementations.



  \subsubsection{struct myint\_t}
  A structure called \emph{myint\_t}, which contains an int array called \emph{nums} of size N and another int variable called \emph{shoot\_value}.
  Due to the nature of the \emph{shoot()} and \emph{target()} functions, the best way to keep track of what value has been shot out is by keeping the shot value (value from the filled nums array picked by random by \emph{shoot()}) tied together with the array it got shot out of. This way, for every int array created, a \emph{shoot\_value} will be associated with it.

\sourcecode{snippets/myint_t.c}{myint\_t implementation}{myint_t}

A helper function was also created to help with the creation of \emph{myint\_t} variables. Really and truely all this function does is return a \emph{myint\_t} variable with its \emph{shoot\_value} set to -1 (-1 will signify a null).

\sourcecode{snippets/myint_t_create_t.c}{Helper function.}{create_t}

  \subsubsection{generate()}
    \sourcecode{snippets/generate_header.c}{Generate Function Prototypes}{generate_header}

\end{onehalfspacing}
% ---------------   End  ----------------




\end{document}